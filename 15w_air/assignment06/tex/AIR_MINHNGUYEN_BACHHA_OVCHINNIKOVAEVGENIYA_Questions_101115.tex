%%%%%%%%%%%%%%%%%%%%%%%%%%%%%%%%%%%%%%%%%
% Short Sectioned Assignment
% LaTeX Template
% Version 1.0 (5/5/12)
%
% This template has been downloaded from:
% http://www.LaTeXTemplates.com
%
% Original author:
% Frits Wenneker (http://www.howtotex.com)
%
% License:
% CC BY-NC-SA 3.0 (http://creativecommons.org/licenses/by-nc-sa/3.0/)
%
%%%%%%%%%%%%%%%%%%%%%%%%%%%%%%%%%%%%%%%%%

%----------------------------------------------------------------------------------------
%	PACKAGES AND OTHER DOCUMENT CONFIGURATIONS
%----------------------------------------------------------------------------------------

\documentclass[paper=a4, fontsize=11pt]{scrartcl} % A4 paper and 11pt font size

\usepackage[T1]{fontenc} % Use 8-bit encoding that has 256 glyphs
\usepackage{fourier} % Use the Adobe Utopia font for the document - comment this line to return to the LaTeX default
\usepackage[english]{babel} % English language/hyphenation
\usepackage{amsmath,amsfonts,amsthm} % Math packages

%\usepackage{lipsum} % Used for inserting dummy 'Lorem ipsum' text into the template

\usepackage{sectsty} % Allows customizing section commands
\allsectionsfont{\centering \normalfont\scshape} % Make all sections centered, the default font and small caps

\usepackage{fancyhdr} % Custom headers and footers
\pagestyle{fancyplain} % Makes all pages in the document conform to the custom headers and footers
\fancyhead{} % No page header - if you want one, create it in the same way as the footers below
\fancyfoot[L]{} % Empty left footer
\fancyfoot[C]{} % Empty center footer
\fancyfoot[R]{\thepage} % Page numbering for right footer
\renewcommand{\headrulewidth}{0pt} % Remove header underlines
\renewcommand{\footrulewidth}{0pt} % Remove footer underlines
\setlength{\headheight}{13.6pt} % Customize the height of the header

\setlength\parindent{0pt} % Removes all indentation from paragraphs - comment this line for an assignment with lots of text

%----------------------------------------------------------------------------------------
%	TITLE SECTION
%----------------------------------------------------------------------------------------

\newcommand{\horrule}[1]{\rule{\linewidth}{#1}} % Create horizontal rule command with 1 argument of height

\title{	
\normalfont \normalsize 
\textsc{Bonn-Rhein-Sieg University of Applied Sciences \\Department of Computer Science} \\ [10pt] % Your university, school and/or department name(s)
\horrule{0.5pt} \\[0.4cm] % Thin top horizontal rule
\LARGE Artificial Intelligence for Robotics - Assignment 06 \\ % The assignment title
\horrule{2pt} \\[0.5cm] % Thick bottom horizontal rule
}
\date{}
\author{Minh Nguyen, Bach Ha \& Evgeniya Ovchinnikova} % Your name
\date{Lecture date: 10 November 2015}
\begin{document}

\maketitle % Print the title

%----------------------------------------------------------------------------------------
%	PROBLEM 1
%----------------------------------------------------------------------------------------

\section{Question}

Describe Hill-climbing algorithm.\\

Hill climbing is an optimization that belongs to local search. Hill climbing is an iterative algorithm that starts in a random local solution and then tries to find a better solution by gradually changing a elements of the solution one by one. If the next local solution is better, the gradually change moves further until no new improvements can be found.\\

It can easily be modified for problems where we require minimization instead of maximization.\\

This algorithm is useful to consider state space landscape.\\

Completeness: "If finitely many local maxima, then $\lim_{restarts \to \infty}$ P (complete) = 1" (from the lecture).

\section{Question}

Name the properties of simulated annealing.\\


1. If the 'temperature' (T) is fixed, probability of $x$ condition is following:\\


$p(x) = \alpha e^{E(x)/kT}$ (Boltzman distribution)\\

2. For small T the ratio between x and x* is the following:\\


$p(x^*)/p(x) = e^{E(x^*)/kT}/e^{E(x)/kT} = e^{(E(x^*)-E(x))/kT}$\\


3. So, if the 'temperature' decreases sufficiently slowly, $Pr[reach x^*] \rightarrow 1$\\

That is why (because of slow decrease of the 'temperature', the algorithm is called 'annealing'.


\section{Question}

Briefly describe local beam search algorithm.\\


Beam search is a heuristic search algorithm that expands the most promising node while searching a graph. For creating a search tree it uses BFS. Basically it is an optimization of BFS that reduces its memory requirements.  It starts with a certain number of randomly generated states. The searchers of the algorithm joining the searcher that has found a good state.\\

Local beam search is a close analogy to natural selection.

\end{document}