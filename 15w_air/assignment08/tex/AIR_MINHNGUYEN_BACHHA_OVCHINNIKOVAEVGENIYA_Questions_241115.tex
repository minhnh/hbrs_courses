%%%%%%%%%%%%%%%%%%%%%%%%%%%%%%%%%%%%%%%%%
% Short Sectioned Assignment
% LaTeX Template
% Version 1.0 (5/5/12)
%
% This template has been downloaded from:
% http://www.LaTeXTemplates.com
%
% Original author:
% Frits Wenneker (http://www.howtotex.com)
%
% License:
% CC BY-NC-SA 3.0 (http://creativecommons.org/licenses/by-nc-sa/3.0/)
%
%%%%%%%%%%%%%%%%%%%%%%%%%%%%%%%%%%%%%%%%%

%----------------------------------------------------------------------------------------
%	PACKAGES AND OTHER DOCUMENT CONFIGURATIONS
%----------------------------------------------------------------------------------------

\documentclass[paper=a4, fontsize=11pt]{scrartcl} % A4 paper and 11pt font size

\usepackage[T1]{fontenc} % Use 8-bit encoding that has 256 glyphs
\usepackage{fourier} % Use the Adobe Utopia font for the document - comment this line to return to the LaTeX default
\usepackage[english]{babel} % English language/hyphenation
\usepackage{amsmath,amsfonts,amsthm} % Math packages
\usepackage{hyperref}

%\usepackage{lipsum} % Used for inserting dummy 'Lorem ipsum' text into the template

\usepackage{sectsty} % Allows customizing section commands
\allsectionsfont{\centering \normalfont\scshape} % Make all sections centered, the default font and small caps

\usepackage{fancyhdr} % Custom headers and footers
\fancyhead[L]{M. Nguyen, B. Ha \& E. Ovchinnikova} % No page header - if you want one, create it in the same way as the footers below
\fancyhead[R]{AIR - Assignment 07 Questions}
\fancyfoot[L]{} % Empty left footer
\fancyfoot[C]{\thepage} % Page numbering for right footer
\renewcommand{\headrulewidth}{0pt} % Remove header underlines
\renewcommand{\footrulewidth}{0pt} % Remove footer underlines
\setlength{\headheight}{13.6pt} % Customize the height of the header

\setlength\parindent{0pt} % Removes all indentation from paragraphs - comment this line for an assignment with lots of text

%----------------------------------------------------------------------------------------
%	TITLE SECTION
%----------------------------------------------------------------------------------------

\newcommand{\horrule}[1]{\rule{\linewidth}{#1}} % Create horizontal rule command with 1 argument of height

\title{	
\normalfont \normalsize 
\textsc{Bonn-Rhein-Sieg University of Applied Sciences \\Department of Computer Science} \\ [10pt] % Your university, school and/or department name(s)
\horrule{0.5pt} \\[0.4cm] % Thin top horizontal rule
\LARGE Artificial Intelligence for Robotics - Assignment 08 Questions \\ % The assignment title
\horrule{2pt} \\[0.5cm] % Thick bottom horizontal rule
}
\author{Minh Nguyen, Bach Ha \& Evgeniya Ovchinnikova} % Your name
\date{Lecture date: 24 November 2015}
\begin{document}

\maketitle % Print the title

%----------------------------------------------------------------------------------------
%	PROBLEM 1
%----------------------------------------------------------------------------------------

\section{Formulate the Minimax Theorem}

Original formulation: every finite, zero-sum, two-person game has optimal mixed strategies. Later was extended to more complex cases of games and other problems, those require making decisions in uncertainty.\\

G - two-person finite zero-sum game with players Max and Min, s*, t* - Min's and Max's strategies correspondingly, $V_G$- G's minimax value:
\begin{itemize}
\item If Min uses t* , Max's expected utility is $\le$ $V_G$ , i.e., $max_s U (s, t*) = V_G$
\item If Max uses s* , Max's expected utility is $\ge$ $V_G$ , i.e., $min_s U (s*, t) = V_G$
\end{itemize}




\section{How to construct a pure strategy for Max?}

\begin{itemize}
\item At each node where it's Max's move, choose one branch
\item At each node where it's Min's move, include all branches
\end{itemize}

\section{Name the properties of alpha-beta algorithm}

\begin{itemize}
\item Pruning does not affect final result
\item Good move ordering improves effectiveness of pruning
\item With "perfect ordering" time complexity = $O(b^{m/2})$
\item A simple example of the value of reasoning about which computations are relevant (a form of metareasoning)
\end{itemize}

\end{document}