%%%%%%%%%%%%%%%%%%%%%%%%%%%%%%%%%%%%%%%%%
% Short Sectioned Assignment
% LaTeX Template
% Version 1.0 (5/5/12)
%
% This template has been downloaded from:
% http://www.LaTeXTemplates.com
%
% Original author:
% Frits Wenneker (http://www.howtotex.com)
%
% License:
% CC BY-NC-SA 3.0 (http://creativecommons.org/licenses/by-nc-sa/3.0/)
%
%%%%%%%%%%%%%%%%%%%%%%%%%%%%%%%%%%%%%%%%%

%----------------------------------------------------------------------------------------
%	PACKAGES AND OTHER DOCUMENT CONFIGURATIONS
%----------------------------------------------------------------------------------------

\documentclass[paper=a4, fontsize=11pt]{scrartcl} % A4 paper and 11pt font size

\usepackage[T1]{fontenc} % Use 8-bit encoding that has 256 glyphs
\usepackage{fourier} % Use the Adobe Utopia font for the document - comment this line to return to the LaTeX default
\usepackage[english]{babel} % English language/hyphenation
\usepackage{amsmath,amsfonts,amsthm} % Math packages
\usepackage{hyperref}

%\usepackage{lipsum} % Used for inserting dummy 'Lorem ipsum' text into the template

\usepackage{sectsty} % Allows customizing section commands
\allsectionsfont{\centering \normalfont\scshape} % Make all sections centered, the default font and small caps

\usepackage{fancyhdr} % Custom headers and footers
\fancyhead[L]{M. Nguyen, B. Ha \& E. Ovchinnikova} % No page header - if you want one, create it in the same way as the footers below
\fancyhead[R]{AIR - Assignment 10 Questions}
\fancyfoot[L]{} % Empty left footer
\fancyfoot[C]{\thepage} % Page numbering for right footer
\renewcommand{\headrulewidth}{0pt} % Remove header underlines
\renewcommand{\footrulewidth}{0pt} % Remove footer underlines
\setlength{\headheight}{13.6pt} % Customize the height of the header

\setlength\parindent{0pt} % Removes all indentation from paragraphs - comment this line for an assignment with lots of text

%----------------------------------------------------------------------------------------
%	TITLE SECTION
%----------------------------------------------------------------------------------------

\newcommand{\horrule}[1]{\rule{\linewidth}{#1}} % Create horizontal rule command with 1 argument of height

\title{	
\normalfont \normalsize 
\textsc{Bonn-Rhein-Sieg University of Applied Sciences \\Department of Computer Science} \\ [10pt] % Your university, school and/or department name(s)
\horrule{0.5pt} \\[0.4cm] % Thin top horizontal rule
\LARGE Artificial Intelligence for Robotics - Assignment 10 Questions \\ % The assignment title
\horrule{2pt} \\[0.5cm] % Thick bottom horizontal rule
}
\author{Minh Nguyen, Bach Ha \& Evgeniya Ovchinnikova} % Your name
\date{Lecture date: 08 December 2015}
\begin{document}

\maketitle % Print the title

%----------------------------------------------------------------------------------------
%	PROBLEM 1
%----------------------------------------------------------------------------------------

\section{Question}

What is an atomic sentence?\\

An atomic sentence (AS) is a declarative sentence, that can not be broken into some other simple sentences and can be either True or False.\\

AS = predicate($term_1 , . . . , term_n$) or $term_1 = term_2$


\section{Question}

How can one create a complex sentence?\\

Complex sentence could be created from atomic or simple sentences by using connectives:

\begin{itemize}
\item $	\neg S$
\item $	S_1 \wedge S_2$
\item $S_1 \vee S_2$
\item $S_1 \Rightarrow S_2$
\item $S_1 \Leftrightarrow S_2$
\end{itemize}


\section{Question}

What is a model in first order logic?\\

It is defined as before: M is a model of a sentence $\alpha$ if $\alpha$ is true in M.
A model is a pair M = (D, I), where D is a domain and I is an interpretation. \\

D contains 1 or more objects (elements of D-domain) and relations between them.\\
I specifies the following references:
\begin{itemize}
\item constant symbols $	\rightarrow$ objects in the domain
\item predicate symbols $	\rightarrow$ relations over objects in the domain
\item function symbols $	\rightarrow$ functional relations over objects in the domain
\end{itemize}

\end{document}