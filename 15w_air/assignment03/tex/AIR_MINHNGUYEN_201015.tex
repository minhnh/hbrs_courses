\documentclass[a4paper, 12pt]{article}
\usepackage{titling}
\usepackage{array}
\usepackage{booktabs}
\setlength{\heavyrulewidth}{1.5pt}
\setlength{\abovetopsep}{4pt}

\setlength{\droptitle}{-7em}

\title{Artificial Intelligence for Robots \\
				- Homework 3 -}
\author{Minh Nguyen}
\date{Lecture date: 20 October 2015}

\begin{document}

\maketitle

\begin{enumerate}

    % Question 1
	\item A finite state space does not always lead to a finite search tree,
    because if the search tree can expand to one of previously expanded state,
    it can loop forever in these states. So a finite state space can have a
    finite state tree if the state space doesn't allow revisiting previously
    reached states, or if the search tree keeps track of and avoids previously
    visited states.

    % Question 2
	\item Implementations of breadth-first and depth-first are included in the
    src/ directory.

	% Question 3
	\item Performance comparison collected from running the program:\\\\
        \begin{tabular}{*7c}
        \toprule
        Category        & \multicolumn{2}{c}{Map 1} & \multicolumn{2}{c}{Map 2} & \multicolumn{2}{c}{Map 3} \\
                        & BFS & DFS                 & BFS & DFS                 & BFS & DFS                 \\
        \midrule
        Elapsed Time(s) & 5   & 3                   & 4   & 3                   & 19  & 18                  \\
        Stored Nodes    & 1261& 674                 & 1076& 824                 & 2460& 1457                \\
        Checked Nodes   & 5033& 3393                & 4277& 3525                & 9841& 9841                \\
        \bottomrule
        \end{tabular}

\end{enumerate}

%\bibliographystyle{plain}
%\bibliography{AIR_MINHNGUYEN_201015}

\end{document}
