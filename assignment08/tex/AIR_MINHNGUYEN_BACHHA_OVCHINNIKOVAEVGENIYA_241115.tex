\documentclass[a4paper, 12pt]{article}
\usepackage{titling}
\usepackage{array}
\usepackage{booktabs}
\usepackage{enumitem}
\usepackage{graphicx}
\setlength{\heavyrulewidth}{1.5pt}
\setlength{\abovetopsep}{4pt}
\graphicspath{{.}}

\usepackage[margin=1in]{geometry}

% Must be after geometry
\usepackage{fancyhdr}
\pagestyle{f    ancy}
\fancyhf{}
\rhead{AIR Homework 8}
\lhead{M. Nguyen, B. Ha \& E. Ovchinnikova}
\cfoot{\thepage}

\setlength{\droptitle}{-5em}

\title{Artificial Intelligence for Robots \\
				- Homework 8 -}
\author{Minh Nguyen, Bach Ha \& Evgenia Ovchinnikova}
\date{Lecture date: 24 November 2015}

\begin{document}

\maketitle

\section{Theoretical questions}

    \begin{itemize}
        \item \emph{What is a solution for a constraint satisfaction problem?}\\
            A solution for a constraint satisfaction problem is a state in which
            all variables are assigned and no constraint is violated.
        \item \emph{What are the components of a constraint satisfaction problem?}
            A constraint satisfaction problem consists of the variables, the
            domain of these variables, and a set of constraints that the values
            assigned to the variables must conform.
        \item \emph{What is a constraint and how is it represented?}
        \item \emph{What is a constraint graph?}
        \item \emph{What are discrete variables and how are they divided?}
        \item \emph{What is preference?}
        \item \emph{What is a successor function?}
        \item \emph{What is constraint propagation?}
        \item \emph{What is arc-consistency?}
        \item \emph{What is Backtracking Search?}
    \end{itemize}

\section{Sudoku puzzle}


\section{Simulated Annealing Implementation}


\end{document}
