\documentclass[a4paper, 12pt]{article}
\usepackage{titling}
\usepackage{array}
\usepackage{booktabs}
\usepackage{enumitem}
\usepackage{graphicx}
\usepackage{hyperref}
\setlength{\heavyrulewidth}{1.5pt}
\setlength{\abovetopsep}{4pt}
\graphicspath{{.}}

\usepackage[margin=1in]{geometry}

% Must be after geometry
\usepackage{fancyhdr}
\pagestyle{fancy}
\fancyhf{}
\rhead{AIR Homework 8}
\lhead{M. Nguyen, B. Ha \& E. Ovchinnikova}
\cfoot{\thepage}

\setlength{\droptitle}{-5em}

\title{Artificial Intelligence for Robots \\
				- Homework 8 -}
\author{Minh Nguyen, Bach Ha \& Evgenia Ovchinnikova}
\date{Lecture date: 24 November 2015}

\begin{document}

\maketitle

\section{Theoretical questions}

    \begin{itemize}
        \item \emph{What is a solution for a constraint satisfaction problem?}\\
            A solution for a constraint satisfaction problem (CSP) is a state in
            which all variables are assigned and no constraint is violated.
        \item \emph{What are the components of a constraint satisfaction problem?}\\
            A constraint satisfaction problem consists of the variables, the
            domain of these variables, and a set of constraints on these variables.
        \item \emph{What is a constraint and how is it represented?}\\
            A constraint specifies the conditions variables must conform to be a
            solution. A constraint can concern value of a single variable (unary)
            or values of a group of variables. A constraint can be absolute or
            by preference. A constraint operated on a discrete, finite domain
            can be represented by enumerating all possible values or combination
            of values.
        \item \emph{What is a constraint graph?}\\
            A constraint graph is a map representation of a binary CSP (where a
            constraint applies to at most two variables). In a constraint graph,
            each node represents a variable, and each edge represents a constraint.
        \item \emph{What are discrete variables and how are they divided?}
        \item \emph{What is preference?}
        \item \emph{What is a successor function?}
        \item \emph{What is constraint propagation?}
        \item \emph{What is arc-consistency?}
        \item \emph{What is Backtracking Search?}
    \end{itemize}

\section{Sudoku puzzle}

\begin{itemize}
    \item Variables: all unknown cells in the 9-by-9 puzzle

	\item Domain: $\{1-9\}$

    \item Constraints
    \begin{itemize}
        \item No same value on a row
        \item No same value in a column
        \item The puzzle is divided into 9 cell groups, each of size 3-by-3. The
        third constraint is no same value in any of these groups.
    \end{itemize}
\end{itemize}

\section{Simulated Annealing Implementation}


\end{document}
