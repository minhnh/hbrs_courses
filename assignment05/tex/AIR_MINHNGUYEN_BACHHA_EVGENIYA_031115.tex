\documentclass[a4paper, 12pt]{article}
\usepackage{titling}
\usepackage{array}
\usepackage{booktabs}
\usepackage{enumitem}
\setlength{\heavyrulewidth}{1.5pt}
\setlength{\abovetopsep}{4pt}

\usepackage[margin=1in]{geometry}

% Must be after geometry
\usepackage{fancyhdr}
\pagestyle{fancy}
\fancyhf{}
\rhead{AIR Homework 5}
\lhead{M. Nguyen, B. Ha \& E. Ovchinnikova}
\cfoot{\thepage}

\setlength{\droptitle}{-5em}

\title{Artificial Intelligence for Robots \\
				- Homework 5 -}
\author{Minh Nguyen, Bach Ha \& Evgenia Ovchinnikova}
\date{Lecture date: 03 November 2015}

\begin{document}

\maketitle

    \begin{enumerate}

    % Question 1
	\item Definitions
	\begin{description}[style=nextline]
        \item [Uniform cost search] A search strategy that always expand the node
            with the lowest path cost on the fringe.
        \item [Depth-first search] A search strategy that always expand the deepest
            nodes, if the nodes are organized into a tree structure.
        \item [Limited depth-first search] A search strategy which performs a depth
            first search until reaching a specific depth limit.
        \item [Iterative deepening search] A limited depth-first search with
            an increasing depth limits.
        \item [Informed search] A search strategy which utilizes provided
            information which cannot be generated systematically from within the
            search algorithm (i.e. path cost, node depth).
        \item [Greedy search] A best first search strategy which uses heuristic
            function(s) to decide which node to be expanded next.
        \item [A* search]
            A best first search that uses the sum of an admissible heuristic
            function and the accumulated path cost as the evaluation function.
        \item [Iterative deepening A* search] Iterative deepening A* performs
            contour-limited A* search with increasing contour sizes. The contour
            size is the upper limit of the expanded nodes' evaluation function
            results.
    \end{description}

    % Question 2
	\item Heuristic Function is a function which estimates the cost from a
    specific node from the goal node.

    \pagebreak
    % Question 3
    \item Implementation of the greedy search can be found in the ``\texttt{src/}'' directory.
        \begin{itemize}
            \item The performance depends on the initial map, though Manhattan
            distance heuristic performs better in general.
            Performance comparison in expanded nodes:

                \begin{tabular}{ccc}
                \toprule
                Configuration               & Manhattan distance    & Misplaced tiles   \\
                \toprule
                $\left[\begin{array}{ccc}
                1 & 4 & 8 \\
                3 & 6 & 2 \\
                0 & 5 & 7
                \end{array}\right]$         & 344                   & 292               \\
                $\left[\begin{array}{ccc}
                1 & 2 & 3 \\
                4 & 5 & 6 \\
                7 & 8 & 0
                \end{array}\right]$         & 342                   & 500               \\
                \bottomrule
                \end{tabular}

            \item As can be seen in the printed heuristics both heuristic functions
            are inconsistent. The heuristics generally decrease but sometimes
            increase from initial state to goal state.
        \end{itemize}

    \end{enumerate}

\end{document}
