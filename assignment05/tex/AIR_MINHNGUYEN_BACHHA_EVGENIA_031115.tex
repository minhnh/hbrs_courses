\documentclass[a4paper, 12pt]{article}
\usepackage{titling}
\usepackage{array}
\usepackage{booktabs}
\usepackage{enumitem}
\setlength{\heavyrulewidth}{1.5pt}
\setlength{\abovetopsep}{4pt}

\setlength{\droptitle}{-7em}

\title{Artificial Intelligence for Robots \\
				- Homework 5 -}
\author{Minh Nguyen \& Bach Ha \& Evgenia Ovchinnikova}
\date{Lecture date: 03 November 2015}

\begin{document}

\maketitle

    \begin{enumerate}

    % Question 1
	\item Definitions
	\begin{description}[style=nextline]
        \item [Uniform Cost Search] A search strategy that always expand the node
            with the lowest path cost on the fringe.
        \item [Depth-First Search] A search strategy that always expand the deepest
            nodes, if the nodes are organized into a tree structure.
        \item [Limited Depth-First Search] A search strategy which performs a depth
            first search until reaching a specific depth limit.
        \item [Iterative Deepening-Search] A limited depth-first search with
            an increasing depth limits.
        \item [Informed Search] A search strategy which utilizes provided
            information which cannot be generated systematically from within the
            search algorithm (i.e. path cost, node depth).
        \item [Greedy search] A best first search strategy which uses heuristic
            function(s) to decide which node to be expanded next.
        \item [A* search]
            A best first search that uses the sum of an admissible heuristic
            function and the accumulated path cost as the evaluation function.
        \item [Iterative deepening A* search] Iterative deepening A* performs
            contour-limited A* search with increasing contour sizes. The contour
            size is the upper limit of the expanded nodes' evaluation function
            results.
    \end{description}

    % Question 2
	\item Heuristic Function is a function which estimates the cost from a
    specific node from the goal node.

    % Question 3
    \item Implementation of the greedy search can be found in the ``\texttt{src/}'' directory.

    \end{enumerate}


\end{document}
