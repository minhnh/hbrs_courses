%!TEX root = ../report.tex

\chapter{Conclusions}


%%%%%%%%%%%%%%%%%%%%%%%%%%%%%%%%%%%%%%%%%%%%%%%%%%%%%%%%%%%%%%%%%%%%%%%%%%%%%%%%
\section{Contributions}

This report reviews recent advances in aspects most relevant to generating data for training a grasp evaluation models,
namely feature extraction from perceptual data, object-grasp representation, grasp evaluation metrics, and data
generation techniques. Additionally, five recent, prominent approaches to data synthesis for grasp evaluation are
examined, and their solutions for each of the four aspects mentioned above are summarized in table
\ref{table:grasp_quality_approaches}. Two of these approaches utilize human labeled data, but their dataset is small
because human grasping datasets are time-consuming and costly to collect. The other three approaches rely on analytical
grasp metrics to label their data, either built into simulators like OpenRAVE or GraspIt!, or calculated directly f
or each grasp candidates. As discussed in section \ref{sub:overview}, analytical metrics have been shown to be
unreliable when applied to real systems, and simulators often fail short in capturing all the details of the real world.

The second contribution of this project is that, in collaboration with another Research and Development project by
Padalkar \cite{Padalkar2018}, a full grasping pipeline is implemented, from perceiving objects to grasp execution. Two
pose estimation methods are implemented, serving as baselines for experimenting and comparing with more advanced grasp
planning techniques.


%%%%%%%%%%%%%%%%%%%%%%%%%%%%%%%%%%%%%%%%%%%%%%%%%%%%%%%%%%%%%%%%%%%%%%%%%%%%%%%%
\section{Future work}

Several extensions and optimizations for the current implementation of the grasping pipeline are possible. First, the
grasp execution implementation can be extended to allow the robot to grasp from different directions. Next, the
detection model for the SSD architecture can also be fine-tuned to perform better in detecting RoboCup@Home objects.
Nearest neighbor algorithms can also be used to improve object pose estimation from the points extracted from RGB-D
clouds. Surface normal calculations can also be implemented to replace the grasp pose calculation provided with the
GQCNN software. Specifically, the pixel coordinate of the detected can be directly transformed to its corresponding 3D
coordinate, and the surface normal around this point can be used as the grasp's approach vector. This surface normal
can also be used as the approach vector for the implemented pose estimation algorithms.

Furthermore, several of the approaches reviewed in chapter 2 can be integrated. Particularly, the shape completion
technique introduced by Varley et al. \cite{Varley2017}, can be used as a prior for Gualtieri et al.
\cite{Gualtieri2016} sampling of the occluded points for their 12-channel 2D representation. Training on the new human
grasp experience dataset by Saudabayev et al. \cite{Saudabayev2018} can also be examined for a direct comparison with
training on data synthesized using analytical grasp metrics and simulation. Techniques to introduce task awareness into
data generation can also be examined, i.e how the knowledge of the task to be performed with the manipulated object can
be encoded into the data synthesis process.
