%!TEX root = ../report.tex

\chapter{Conclusions}


%%%%%%%%%%%%%%%%%%%%%%%%%%%%%%%%%%%%%%%%%%%%%%%%%%%%%%%%%%%%%%%%%%%%%%%%%%%%%%%%
\section{Contributions}
\begin{itemize}
    \item Review of recent approaches to synthesizing data to train a grasp evaluation model
    \item Integration of the full grasp pipeline on the HSR, tested with two baseline grasp planning approaches using
    pose estimation. More advanced grasp planning methods can now be examined for comparison.
\end{itemize}


%%%%%%%%%%%%%%%%%%%%%%%%%%%%%%%%%%%%%%%%%%%%%%%%%%%%%%%%%%%%%%%%%%%%%%%%%%%%%%%%
\section{Lessons learned}


%%%%%%%%%%%%%%%%%%%%%%%%%%%%%%%%%%%%%%%%%%%%%%%%%%%%%%%%%%%%%%%%%%%%%%%%%%%%%%%%
\section{Future work}
\begin{itemize}
    \item Extend grasp execution to allow the robot to grasp from different directions.
    \item Retrain detection models with RoboCup@Home objects
    \item Use nearest neighbor algorithms on the extracted points for better pose estimation
    \item Add surface normal calculation from the extracted points and combine with GQCNN to redo their grasp pose
    calculation, using only the angle of the gripper in the image plane and the gripper center's pixel coordinates
    \item Integrate shape completion approach from Varley et al. \cite{Varley2017} for better pose estimation, combine
    with object-grasp representation from Gualtieri et al. \cite{Gualtieri2016}, train on new human grasp database
    introduced by Saudabayev et al. \cite{Saudabayev2018}.
\end{itemize}