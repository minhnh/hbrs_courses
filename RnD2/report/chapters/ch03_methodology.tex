%!TEX root = ../report.tex
\chapter{Methodology}

\picHereWidth{grasp_pipeline}{Pipeline for object grasping.}{fig:grasp_pipeline}{\textwidth}

In order to perform the experiments described in the next chapter, it is necessary to integrate a full object grasping
pipeline. Figure \ref{fig:grasp_pipeline} illustrates the main components of this pipeline. Details on the approaches
considered and implemented for each component will be discussed in this chapter.


%%%%%%%%%%%%%%%%%%%%%%%%%%%%%%%%%%%%%%%%%%%%%%%%%%%%%%%%%%%%%%%%%%%%%%%%%%%%%%%%
\section{Object detection}
\picHereWidth{bad_objects}{Example of objects that are partially invisible in point clouds.}
             {fig:bad_objects}{\textwidth}

Originally object detection in the Robocup@Home team's code base \footnote{GitHub link for the source code:
\url{https://github.com/b-it-bots/mas\_perception/}} is done using existing algorithms available in the
\footnoteHref{http://docs.pointclouds.org/}{Point Cloud Library (PCL)}. First, a random sample consensus (RANSAC)
algorithm is executed to find a plane and its inlaying points using point clouds from the RGB-D camera (multiple clouds
are accumulated before processing to alleviate noise from the sensor). Only planes whose normals are parallel to the
$ z $-axis of the world frame, or in other words, parallel to the ground, are considered. The convex hull of the plane
and a given height are then used to form a 3D polygonal prism and extract points inside this prism. Clusters are then
segmented from the extracted points as objects. In practice, this method proves to be unreliable for Robocup@Home
objects as the SAC algorithm does not always find a plane, especially when several objects are present. Relying solely
on the depth information can also prove inadequate for object detection, as some materials of domestic objects can be
noisy or even invisible for the depth sensor (two examples of which are shown in figure \ref{fig:bad_objects}). The 26
parameters required for tuning the detection pipeline also make it inflexible in adapting different surfaces and
objects. Finally, execution of the algorithm is slow, taking several seconds for plane detection and cluster
extraction.

The HSR from Toyota provides a built-in segmentation solution based on an existing
\footnoteHref{http://www.ros.org}{Robot Operating System (ROS)} package called \linebreak
\footnoteHref{http://wiki.ros.org/tabletop\_object\_detector}{\texttt{tabletop\_object\_detector}}. ROS is a software
framework designed for robots, containing drivers for various robotic hardware devices as well as libraries commonly
used in robotic applications. The detection package assumes objects to have a rotationally symmetric shape, to have a
fixed orientation on a horizontal surface (e.g. a table), and to be no more than 3cm apart from each other. This
package follows a similar segmentation algorithm to the aforementioned method, first detecting a plane then clustering
points above this plane as objects. While the package is better optimized and performs faster than the previous
implementation, the source code of the modified version is not released by Toyota, and the algorithm suffers from
limitations similar to the ones described above.

%%%%%%%%%%%%%%%%%%%%%%%%%%%%%%%%%%%%%%%%%%%%%%%%%%
\subsection{Single Shot MultiBox Detector}

\picHereWidth{liu_et_al-2016-ssd_arch}{SSD architecture \cite{Liu2016SSD}.}{fig:ssd_arch}{\textwidth}

Because the above detection implementations prove not reliable enough for performing grasp experiments, the Single Shot
MultiBox Detector (SSD) algorithm \cite{Liu2016SSD} was integrated into the grasping pipeline for this project. In
recent years, the success of CNN in image recognition and the introduction of large-scale object detection datasets
(i.e. \footnoteHref{http://cocodataset.org/}{Microsoft COCO} and
\footnoteHref{http://host.robots.ox.ac.uk/pascal/VOC/voc2012/}{Pascal VOC}) has bolstered research interests in
detecting objects directly from RGB images \cite{Gu2018}. Gu et al. \cite{Gu2018} include a review of recent CNN-based
object detection architectures in their survey of CNN. Among these approaches, SSD stands out as being able to detect
objects accurately at a high frame rate. The architecture introduces a set of ``default boxes'' for each training image
and select ones which has a certain amount of overlapping with the ground truth boxes. The network then learns the
offsets between the ground truth and default boxes, specifically between their center coordinates, widths, and heights,
alongside with a confidence score for the object class detected. For $ N $ selected default boxes, the overall loss
function is the sum of the localization loss ($ L_{loc} $) and confidence loss ($ L_{conf} $):
\[ L(x,c,l,g) = \cfrac{1}{N} \left( L_{conf}(x, c) + L_{loc}(x,l,g) \right) \],
where $ x $ is a chosen default box, $ c $ is the predicted confidence, $ l $ is the predicted box, and $ g $ is the
ground truth box. Detection training is applied to multiple resolutions of the feature map (illustrated in figure
\ref{fig:ssd_arch}) for the network to learn scale awareness.

\picHereWidth{image_detection_class_structure}
             {Class diagram for the image detection implementation. In italics are abstract classes and methods.}
             {fig:detection_class_diagram}{\textwidth}

The SSD implementation by Pierluigi Ferrari \footnote{GitHub link for source code:
    \url{https://github.com/pierluigiferrari/ssd_keras}} is extended to work with ROS and the Robocup@Home perception
software structure. To allow for easy adaptation to different image detection architectures and implementations in the
future, an abstraction layer for the image detection service was created, as illustrated in figure
\ref{fig:detection_class_diagram}. The \texttt{ImageDetectionService} class loads the specific realization of
\texttt{ImageDetector} class and its configurations at launch via constructor parameters and uses its instantiation to
handle detection service requests from other ROS nodes during runtime. Figure \ref{fig:detection_example} shows a
sample detection result from the integrated SSD architecture. For detecting RoboCup@Home objects, we use a provided
model trained on the COCO dataset.

\picHereWidth{detection_example}
             {Example of SSD detection using a model trained on the COCO dataset. The bounding box label contains the
              detected object class and the prediction confidence.}
             {fig:detection_example}{0.7\textwidth}


%%%%%%%%%%%%%%%%%%%%%%%%%%%%%%%%%%%%%%%%%%%%%%%%%%%%%%%%%%%%%%%%%%%%%%%%%%%%%%%%
\section{Pose estimation and grasp planning}

This section describes how grasp poses are inferred from the objects detected in RGB images. First, simple pose
estimation algorithms are implemented as the baseline grasp planning method. Next, the GQCNN grasp planner introduced
by Mahler et al. \cite{mahler2017} is integrated for comparison.

\picHereWidth{base_link_frame}{\texttt{base\_link} coordinate frame. The $ z $-axis points upward into the robot.}
             {fig:base_link_frame}{0.5\textwidth}

%%%%%%%%%%%%%%%%%%%%%%%%%%%%%%%%%%%%%%%%%%%%%%%%%%
\subsection{Pose estimation for detected objects} \label{sub:pose_estimation}

\picHereWidth{grasp_plan_pose_estimation}{Flowchart for the pose estimation pipeline.}
             {fig:grasp_plan_pose_estimation}{\textwidth}

For the baseline method, the grasp approach vector is assumed to align with the $ x $-axis of the robot base coordinate
frame, as illustrated in figure \ref{fig:base_link_frame}. This assumption reduces the 6-DOF grasp-pose-finding problem
to finding the 3D coordinates of the object. As shown in figure \ref{fig:grasp_plan_pose_estimation}, the RGB image is
extracted from the point cloud for detection. After the image detection service responded with 2D bounding boxes of the
objects, point clouds are transformed to the robot's base frame, and points within the detected regions are extracted
from the transformed clouds. This step relies on the second assumption that the point clouds are organized: the points
resemble an image or a 2D matrix where data is organized into rows and columns. This allows direct projection of pixel
coordinates to points in the point cloud. As PCL is implemented in \texttt{C++}, while the SSD implementation is
written in \texttt{Python}, interfaces between the two languages are developed for functions processing point clouds
using the \footnoteHref{https://www.boost.org/doc/libs/1\_68\_0/libs/python/}{\texttt{Boost.Python}} library. The
grasping position $ \mathbf{p} = (x, y, z) $ is calculated from the extracted 3D coordinates. If matrix $ B $ of shape
$ M \times 3 $ contains all the 3D coordinates extracted using the detected object's bounding box, we either take the
closest point along the $ x $-axis of the base frame:
\begin{equation} \label{eq:pose_estimation_min}
    p = \left( \begin{matrix}
    x \\ y \\ z
    \end{matrix} \right) =
    \left( \begin{matrix}
    \min_{i = 1}^M B_{i,1} \\
    \cfrac{1}{M} \sum_{i = 1}^{M} B_{i,2} \\
    \cfrac{1}{M} \sum_{i = 1}^{M} B_{i,3}
    \end{matrix} \right)
\end{equation}
, or the mean along all three axes:
\begin{equation} \label{eq:pose_estimation_mean}
    p_j = \cfrac{1}{M} \sum_{i = 1}^{M} B_{i,j}
\end{equation}
, where $ j = 1, 2, 3 $.

%%%%%%%%%%%%%%%%%%%%%%%%%%%%%%%%%%%%%%%%%%%%%%%%%%
\subsection{Grasp detection with GQCNN}

\picHereWidth{grasp_plan_gqcnn}{Flowchart for the GQCNN pipeline.}
             {fig:grasp_plan_gqcnn}{0.55\textwidth}

The GQCNN grasp planner introduced by Mahler et al. \footnote{GitHub link:
\url{https://github.com/BerkeleyAutomation/gqcnn}} \cite{mahler2017} is integrated into ROS as a grasp detection
service, which takes in a color, a depth image, and the detected object's bounding box. After planning it returns a
pregrasp pose and a probability of grasp success. The planner first combines the image pair into a single 4-channel
matrix and uses the 2D bounding box to crop out the relevant region. Bipedal grasp candidates, characterized by the
pixel coordinates of the grasp center and the gripper's angle in the image plane, are then sampled for the cropped
matrix. Corresponding local representations (as described in section \ref{subsub:object_grasp_local}) are extracted for
each grasp as input for the GQCNN model. The output success probability scores are compared to find the optimal grasp.
To calculate the 3D grasp pose, the provided planner assume that the approach vector to align with the camera axis. The
returned pose is 10cm away from the object's surface along this axis.

\picHereWidth{grasp_gqcnn_result}
             {The blue arrow represents an example grasp pose returned from the GQCNN grasp planner. The number
              in parentheses is the probability of grasp success returned by the GQCNN model. On the left is the
              visualized object detection result.}
             {fig:grasp_gqcnn_result}{\textwidth}

For this project, the GQCNN model trained on the full Dex-Net 2.0 dataset containing 6.7 million synthetic data points
\cite{mahler2017} is integrated. Figure \ref{fig:grasp_gqcnn_result} shows an example of grasp poses returned from the
grasp planner using this model. Out of the four objects selected for experiments (shown in figure \ref{fig:objects}),
the integrated planner was only able to detect grasps for the noodle box at very low confidence. The planner's
unreliability in grasp detection and its long planning time (from 10 to 20 seconds per detected object) discourage from
including them in the grasp experiments performed. This can be attributed to the experimental setup, which is different
from the statically mounted RGB-D camera with its principal axis perpendicular to the grasping surface (e.g. a table)
as
described in the original paper. However, for a mobile robot like the HSR, such a setup is often impractical.


%%%%%%%%%%%%%%%%%%%%%%%%%%%%%%%%%%%%%%%%%%%%%%%%%%%%%%%%%%%%%%%%%%%%%%%%%%%%%%%%
\section{Grasp execution}

Originally manipulation planning is done using the \footnoteHref{http://moveit.ros.org}{MoveIt!} mobile manipulation
library. Currently, MoveIt! only fully supports planners available in the \footnoteHref{http://ompl.kavrakilab.org}{
Open Motion Planning Library (OMPL)}, which are primarily randomized planners. This kind of planners samples a set of
joint trajectories and choose the first plan that is physically feasible. Because of their stochastic nature, the
planners may give unnatural, unexpected manipulation plans, such as one shown in figure \ref{fig:grasp_moveit_fail}.

\picHereWidth{grasp_moveit_fail}{MoveIt! failing to plan a simple grasp.}
             {fig:grasp_moveit_fail}{\textwidth}

In order to have a more reliable setup for grasp experiments, manipulation motion is planned using the integration and
extension of Dynamic Motion Primitives (DMP) libraries for a parallel project by Padalkar \cite{Padalkar2018}. DMP
models demonstrated trajectories as a series of differential equations, allowing the robot to learn human motions and
calculate kinematic commands via solving these equations to perform new manipulation tasks in a similar way. The work
by Padalkar extends the DMP libraries to also use navigation commands to perform manipulation tasks that can't be
reached by the learned motion alone.
