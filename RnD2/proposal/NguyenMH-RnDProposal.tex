\documentclass[12pt]{article}

%-------------------------------------------------
%   THEMES & PACKAGES
%-------------------------------------------------
\usepackage{fancyhdr}
\usepackage{lastpage}
\usepackage{xcolor}
\usepackage{pdfpages}
\usepackage{hyperref}
\usepackage{graphicx}
\usepackage{enumitem}
\usepackage{todonotes}

%%% Custom colors & commands
\newcommand{\HRule}[1]{\rule{\linewidth}{#1}}   % Horizontal rule

%-------------------------------------------------
%   COMMON INFO
%-------------------------------------------------
\newcommand{\hmwkTitle}{R \& D Proposal}
\newcommand{\hmwkTopic}{Transfer Learning for Object Grasping}
\newcommand{\hmwkDueDate}{December 15, 2017}
\newcommand{\hmwkClass}{Research and Development}
\newcommand{\hmwkAuthorName}{Minh H. Nguyen}
\newcommand{\hmwkAuthorSchool}{Bonn-Rhein-Sieg University of Applied Sciences}
\newcommand{\hmwkAdvisorFirst}{Prof. Dr. Paul G. Pl\"{o}ger}
\newcommand{\hmwkAdvisorSecond}{Alex Mitrevski}
\newcommand{\hmwkAdvisorThird}{Maximilian Sch\"{o}bel}

\graphicspath{{../images/}}

%-------------------------------------------------
%   MARGINS
%-------------------------------------------------
\topmargin=-0.75cm
\evensidemargin=0cm
\oddsidemargin=0cm
\textwidth=16.0cm
\textheight=22.0cm
\headsep=0.6cm

%-------------------------------------------------
%   HEADERS & FOOTERS
%-------------------------------------------------
\pagestyle{fancy}
%-------------------------------------------------
% Special first page
\fancypagestyle{firststyle} {
    \fancyhf{}
    \fancyfoot{}
    \renewcommand\headrulewidth{0.pt}
    \renewcommand\footrulewidth{0.pt}
}
%-------------------------------------------------
\lhead{\hmwkAuthorName}
\rhead{\hmwkTitle}
%-------------------------------------------------
\cfoot{Page \thepage\ of \protect\pageref{LastPage}}
%-------------------------------------------------
\renewcommand\headrulewidth{0.4pt}
\renewcommand\footrulewidth{0.4pt}

%-------------------------------------------------
%   TITLE
%-------------------------------------------------
\title{\normalsize \textsc{\hmwkAuthorSchool}   % Subtitle
    \\[2.0cm]                                   % 2cm spacing
    \HRule{0.5pt} \\                            % Upper rule
    \LARGE \textbf{\uppercase{\hmwkTopic}}
    \HRule{2pt} \\ [0.5cm]                      % Lower rule + 0.5cm spacing
    \hmwkTitle\\[0.5cm]
    \normalsize \hmwkDueDate\\
}

\author{
    \\[4.0cm]
    Author:\\
    \hmwkAuthorName\\
    \ \\
    Advisors:\\
    \hmwkAdvisorFirst\\
    \hmwkAdvisorSecond\\
    \hmwkAdvisorThird\\
}

\date{}

%-------------------------------------------------
%   BIBLIOGRAPHY
%-------------------------------------------------
\usepackage[english]{babel}
\usepackage[backend=biber]{biblatex}
\addbibresource{../RnD.bib}
%\addbibresource{proposal.bib}

%-------------------------------------------------
%   BEGIN
%-------------------------------------------------
\begin{document}
%-------------------------------------------------
    \maketitle
    \thispagestyle{firststyle}
    \newpage

%-------------------------------------------------
\section{Introduction}

    %---------------------------------------------
    \subsection{Motivation}
    \begin{itemize}
        \item Grasp planning is generally divided into analytical and empirical approaches \cite{Bohg2014}.
        \item Analytic methods often optimize some metrics which measure one or multiple properties of a robotic hand, namely dexterity, equilibrium, stability or ability to exhibit a certain dynamic behavior \cite{Bohg2014}. However, these methods often make assumptions about the object (i.e. shapes or poses) and therefore are prone to errors and does not generalize well to novel objects, while taking more time to match point clouds to known models \cite{Goldfeder2011}.
        \item Empirical or data-driven approaches generally sample from several grasp candidates and rank them according to a specific metric \cite{Bohg2014}. \todo{problem with these approaches}
        \item Mahler et al. \cite{mahler2017} proposed Grasp Quality Convolutional Neural Network (GQ-CNN) as a model to rapidly predict probability of successful grasp from depth image. Despite achieving high precision in successful grasp prediction, the model is limited in how grasps are represented: the grasp's approach vector and camera view point is presumed to be perpendicular to the table, while the gripper configuration and wrist orientation used as training input are simplified to the angle of the parallel-jaw grasp axis with respect to the table. The model is therefore insufficient in generalizing for use with mobile robots, where the camera pose often changes as the robot moves, and the grasp approach vector is not known beforehand.
        \item This research aims to extend the work by Mahler et al. \cite{mahler2017} for use with mobile robotics, specifically on the Care-O-bot 3 platform \todo{COB link footnote}, through synthesizing a new dataset using a more general grasp representation and training a new CNN model on this dataset. \todo{align motivation with project description slides}
    \end{itemize}


    %---------------------------------------------
    \subsection{Prior Work}
    \begin{figure}[h!]
        \centering
        \missingfigure{mind map}
        \caption{TODO mind map}
        \label{fig:mindmap}
    \end{figure}

	\subsubsection{Grasp representation}
	Bohg et al. \cite{Bohg2014} parameterize grasps as
	\begin{itemize}
		\item the \emph{grasping point} of the object where the gripper should be aligned,
		\item an \emph{approach vector} from which the gripper shall approach the \emph{grasping point},
		\item the \emph{wrist orientation} of the robotic hand,
		\item and an \emph{initial gripper configuration}.
	\end{itemize}

    \subsubsection{Grasp quality evaluation}

    \subsubsection{Convolutional Neural Network as feature extractor for point clouds}

    \subsubsection{Dex-Net 2.0 Data Generation}

    %---------------------------------------------
    \subsection{Approach}
	\begin{itemize}
		\item define grasp plan representation in 3D.
		\item generate synthesis point cloud data based on this representation following the Dex-Net 2.0 data generation process \cite{mahler2017}
	\end{itemize}

    %---------------------------------------------
    \subsection{Expected Results}


%-------------------------------------------------
\section{Project Plan}

    %---------------------------------------------
    \subsection{Work Packages}

    %---------------------------------------------
    \subsection{Work Schedule}
    \missingfigure{TODO Grannt chart}

\printbibliography

%-------------------------------------------------
%   END
%-------------------------------------------------
\end{document}
