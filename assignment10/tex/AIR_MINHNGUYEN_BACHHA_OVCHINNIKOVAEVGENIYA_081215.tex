\documentclass[a4paper, 12pt]{article}
\usepackage{titling}
\usepackage{array}
\usepackage{booktabs}
\usepackage{enumitem}
\usepackage{graphicx}
\usepackage{hyperref}
\setlength{\heavyrulewidth}{1.5pt}
\setlength{\abovetopsep}{4pt}
\graphicspath{{.}}

\usepackage[margin=1in]{geometry}

% Must be after geometry
\usepackage{fancyhdr}
\pagestyle{fancy}
\fancyhf{}
\rhead{AIR Homework 9}
\lhead{M. Nguyen, B. Ha \& E. Ovchinnikova}
\cfoot{\thepage}

\setlength{\droptitle}{-5em}

\title{Artificial Intelligence for Robots \\
				- Homework 9 -}
\author{Minh Nguyen, Bach Ha \& Evgenia Ovchinnikova}
\date{Lecture date: 1 December 2015}

\begin{document}

\maketitle

\section{Theoretical questions}
    \begin{itemize}
        \item \emph{What is a strategy?}\\
            A strategy is a set of rules that specifies an agent's action in every possible situation.
        \item \emph{What is the goal of Max agent?}\\
        	The goal of Max agent is to maximize the Max’s expected utility.
        \item \emph{What is the goal of Min agent?}\\
        	The goal of Min agent is to minimize the Max’s expected utility.
        \item \emph{What is the Minimax Theorem?}\\
        	Minimax theorem is about a two-person finite zero-sum game G. The two players are Min and Max.
        		There are two strategies s* and t* and the game G's minimax value: $V_G$.
        	\begin{itemize}
        		\item If Min uses t*, Max's expected utility would be smaller or equal to $V_G$.
        		\item If Max uses s*, Max's expected utility would be greater or equal to $V_G$.
			\end{itemize}      
		\item \emph{How do we construct an strategy for Max?}\\
			At the node, where it is Max's turn, chooses one branch. At the node, where it is Min's turn, 
			takes all branchs into consideration.
		\item \emph{How do we construct an strategy for Min?}\\
			At the node, where it is Min's turn, chooses one branch. At the node, where it is Max's turn, 
			takes all branchs into consideration.	
		\item \emph{How do we find the best strategy?}\\
			Brute-force algorithm, Minimax algorithm or alpha-beta pruning algorithm can be used to find the
			best strategy.
		\item \emph{How does the Minimax algorithm work?}\\		      
			Agents decide action using the minimax value of the node. In minimax algorithm, the minimax value of a node
			is calculated from the its branch. When it is Max's turn, the minimax value of a node is the highest minimax 
			value of its branchs. During Min's turn, the minimax value of a node is the smallest minimax value of its branchs.
		\item \emph{Is the Minimax algorithm affordable for chess? Why?}\\
			No. Because the branching factor ($b \approx 35$) and max-depth ($b \approx 100$) of chess are too high, 
			there would be around $35^{100}$ nodes, thus the search would take an unaffordable amount of time.
		\item \emph{What is alpha-beta pruning?}\\
			Alpha-beta pruning means stop computing the value of branchs in which, the value is worse
			than the current best founded value.
		\item \emph{Is the alpha-beta pruning affordable for chess? Why?}\\
			No. Because even with the best move ordering, the time complexity is still $35^{50}$.
    \end{itemize}

\end{document}
