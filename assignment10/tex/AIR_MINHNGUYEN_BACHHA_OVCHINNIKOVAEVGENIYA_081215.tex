\documentclass[a4paper, 12pt]{article}
\usepackage{titling}
\usepackage{array}
\usepackage{booktabs}
\usepackage{enumitem}
\usepackage{graphicx}
\usepackage{hyperref}
\setlength{\heavyrulewidth}{1.5pt}
\setlength{\abovetopsep}{4pt}
\graphicspath{{.}}

\usepackage[margin=1in]{geometry}

% Must be after geometry
\usepackage{fancyhdr}
\pagestyle{fancy}
\fancyhf{}
\rhead{AIR Homework 10}
\lhead{M. Nguyen, B. Ha \& E. Ovchinnikova}
\cfoot{\thepage}

\setlength{\droptitle}{-5em}

\title{Artificial Intelligence for Robots \\
				- Homework 10 -}
\author{Minh Nguyen, Bach Ha \& Evgenia Ovchinnikova}
\date{Lecture date: 8 December 2015}

\begin{document}

\maketitle

\section{Theoretical questions}
   
\subsection{What is a Knowledge-Based Agent (KBA)}

Knowledge-Based Agent (KBA)is an agent, that keeps track of the world in an internal state, that KBA regularly updates. The agent must include a knowledge base (set of sentences in a formal language) and a system of inference. KBA should be capable of the following:

\begin{itemize}
\item Having a knowledge about the world, in which it is acting
\item Representing states, actions, etc.
\item Assimilating new percepts
\item Updating internal state, that represents the world
\item Deducing hidden properties of the world
\item Reasoning about the appropriate course of actions
\end{itemize}


\subsection{What is Knowledge Base (KB)?}

Knowledge Base (KB) is a collection of sentences in a formal language.

\subsection{What are the two operations that an KBA can perform?}

KBA can perform the following operations:

\begin{itemize}
\item KBA can form a representation of the world, in which it is acting
\item KBA can create a new representation of the world by using a system of inference and then use this new representation to decide what to do next
\end{itemize}


\subsection{What is the characteristic of logical reasoning?}
Logical reasoning is a process of making conclusion by thinking in a logical way.
The main characteristics of logical reasoning are its persuasiveness and conclusiveness.

\subsection{What is syntax ?}

Syntax is a set of the rules that governs the structure of the sentences in the language.

\subsection{When can we say that m is a model of sentence $\alpha$ ?}

Model is a formally structured world where the evaluation of the truth is possible, i.e. to
each symbol of the world a (True/False) value is assigned. So, if a certain knowledge base contains k symbols, 2k models are possible.\\
So, m is a model of sentence $\alpha$ if $\alpha$ is true in m.

\subsection{What is entailment ?}

Entailment is a semantic-based relationship between sentences (syntax). It means that one thing follows from the other, i.e. knowledge base entails a sentence (S) if and only if the sentence is True and all worlds where the knowledge base is True: KB $|$= S.

\subsection{What is inference ?}

An algorithm of inference allows to derive a sentence (S) from the knowledge base (KB) by a certain procedure (i):  $KB |=_i S$ . If the algorithms derives only the sentences, those are entailed by the knowledge base, i is a sound. i is complete if the following requirement is met: whenever KB $|$= S, $KB |=_i S$ as well.


\subsection{What is the difference between entailment and inference?}

It could be explained with a following example: if knowledge base is considered as a haystack, S could be considered as a needle. Then entailment is this needle and inference is an algorithm of finding it.


\section{Minimax and Alpha-beta pruning}

Alpha-beta is expected to have less space requirement and run faster than regular minimax,
as not all nodes need to be visited. Minimax requires a depth limit to be run efficiently.

\end{document}
